% !TeX spellcheck = en_US
% !TeX program = xelatex
\documentclass[12pt]{article}
\usepackage{npu-report-style}
\setmainfont{Times New Roman}

\usepackage{enumerate}

% 显示参考文献、引用超链接
\usepackage[breaklinks,colorlinks,linkcolor=black,citecolor=black,urlcolor=black]{hyperref}
% 关闭参考文献、引用超链接
%\usepackage[pagebackref=false, colorlinks, bookmarks=false ]{hyperref}

\usepackage{graphicx, subfig, epstopdf, amsmath, amssymb,booktabs, paralist, float, caption, xfrac, enumitem, array, multirow, cite}
\usepackage{textcomp,mathcomp}%摄氏度等符号 \textcelsius 
%\usepackage{graphicx, subfig, epstopdf, amsmath, booktabs, paralist, float, caption, xfrac, enumitem, array, multirow}
\graphicspath{{./figure/}}

%caption setting
%\usepackage{caption}
%\captionsetup[table]{labelsep=space} % 表
%\captionsetup{labelsep=none} % 图 
%\usepackage{subcaption}

%\usepackage{subfigure}
\usepackage{subfloat}

\usepackage[heading=true]{ctex}

\usepackage{zhnumber} % change section number to chinese
\renewcommand\thesection{\zhnum{section}}
\renewcommand \thesubsection {\arabic{section}}
%\renewcommand \thesubsection {\arabic{subsection}}

%\usepackage{bibspacing}
%\setlength{\bibspacing}{\baselineskip}

\usepackage{url}
%\usepackage{cite}
% comma: 用逗号分隔多个引用; square:使用方括号; super:引用是上角标形式
\usepackage[square,sort&compress,super,numbers]{natbib}

%调整参考文献之间的距离(类似行距)
%\setlength{\bibsep}{9pt}

%加入下面这一行,非常灵活。 其中第一个参数是字体大小, 第二个参数是行距大小。自己可以视情况调整。
%\def\bibfont{\fontsize{10.5pt}{12}\selectfont}
\usepackage{setspace}
\linespread{1.5}
%\usepackage{natbibspacing}

%\usepackage{subfigure}
\usepackage{titlesec}
%\titleformat{\section}{\Large\bfseries}{\thesection、}{0em}{}
%\titleformat{\section}{\heiti\Large}{\thesection、}{0em}{}
%\titleformat{\subsubsection}{\kaishu\bfseries}{\thesection}{0em}{}

\titleformat{\section}{\heiti\zihao{-4}}{\thesection、}{0em}{}
\titleformat{\subsubsection}{\kaishu\bfseries\zihao{-4}}{\thesection}{0em}{}

\renewcommand{\refname}{\textbf{1.3参考文献} } % 修改参考文献部分的标题
%anonymous_review
%\def\anonymous{yes}

%\usepackage[heading=true]{ctex}

%\usepackage{gbt7714}

%\renewcommand{\bibname}{Reference}


\DeclareCaptionLabelSeparator{twospace}{\ ~}   %这三条语句即可
\captionsetup[figure]{labelsep=space,name={\kaishu 图},font=small,justification=centering}
%\captionsetup{labelsep=period}
%\captionsetup[figure]{name={Fig.},labelsep=period} 
%space去掉点
%period加点
%不加space、period这两个就是冒号
\setlength{\abovecaptionskip}{0.2cm}
\setlength{\belowcaptionskip}{-0.5cm}

% \newitemsep
% 下划线
\makeatletter
\newcommand\dlmu@underline[2][6cm]{%
	\hskip1pt\underline{\hb@xt@ #1{\hss#2\hss}}\hskip3pt}
\let\coverunderline\dlmu@underline
\makeatother


\ifx\anonymous\undefined
\def\author{ 张\quad 三 }
\def\school{ 机电工程学院}
\def\mymajor{ 机械制造及其自动化}
\def\idnumber{ xxxxxxxxxxxx} %自己的学号
\def\supervisedsir{王五}
\def\anonymousSupervisedsir{}
\def\sirposition{ \ 教授}
%\def\sirposition{\  教授}
%\def\authorEnglishName{\textbf{S. Zhang}$^*$}
%\def\supervisedsirEnglishName{\textbf{Z. Xue}$^\dag$}
\def\labName{XXX实验室}
\def\subjectName{XXX}
\else
	\def\author{}
	\def\school{}
	\def\mymajor{}
	\def\idnumber{}
	\def\supervisedsir{}
	\def\sirposition{}
	\def\anonymousSupervisedsir{***}
%	\def\authorEnglishName{*}
%	\def\supervisedsirEnglishName{\dag}
	\def\labName{***}
	\def\subjectName{***}
\fi

%\renewcommand\figureautorefname{图}		% 重新定义引用图标
%\renewcommand\tableautorefname{表}		% 重新定义应用表格
%\def\equationautorefname{式}				% 重新定义公式

% 设置 `论文题目' 
\nputitle{如何排版自己的博士论文}

\npunumber{\idnumber}
\npuschool{\school}
\npumajor{\mymajor}
\npuname{\author}
\npudegree{博士}
\npusupervisor{ \supervisedsir \sirposition}
\nputype{全日制学术型}
\npudate{ 2022/12/20}
%\npudate{\today}
% 博士
\npucheckphdtrue
% 第一次
\npucheckfirsttrue
%\npuchecksecondtrue

%% 设置 `论文类型': 取消或添加注释即可勾选相应类型
\npucheckbasetrue   	%  基础研究
%\npucheckapplytrue     	%  应用研究
%\npucheckenginetrue   	%  工程技术
%\npucheckoveralltrue   %  跨学科研究

\begin{document}
\section{学位论文研究依据}
\noindent{\zihao{-4}\kaishu 学位论文的选题依据和研究意义,国内外研究现状和发展态势,主要参考文献,以及已有的工作积累和研究成果。}


\subsubsection*{1.1选题背景和研究意义}

为什么要做这件事情?

%\vspace{-0.2cm}
\subsubsection*{1.2国内外研究现状}

国内外相关领域对这件事情的类似研究现状,分类叙述。

\subsubsection*{1.2.1 xxxxxx}

李四\cite{zhang2022}干了什么事情,取得了什么成果。

\begin{figure}[htb]
	\centering
	%\includegraphics{go-rf.png}
	\includegraphics[width=0.4\linewidth]{nwpu-logo.png}
	\caption{西北工业大学logo}
	\label{fig-0101}
\end{figure}

\begin{equation*}
	\nabla p = -\frac{\mu}{K} \vec{v} 
\end{equation*}

\subsubsection*{1.2.2 xxxxxxxx}





%\vspace{0.5cm}
%\subsubsection*{1.2.1 XXXXXX}

\begin{spacing}{1.2}
		\small
		%\bibliographystyle{nputhesis}          % 参考文献格式
		\bibliographystyle{gbt7714-numerical}
		\bibliography{ref}      % expects file "reference.bib
\end{spacing}


%\subsubsection*{1.3存在问题及展望}


\subsubsection*{1.4已有工作积累与研究成果}


\clearpage
\newpage

\section{学位论文研究内容}
\noindent{\kaishu 学位论文的研究目标、研究内容及拟解决的关键性问题(可续页)}
\subsubsection*{2.1论文的研究目标}

通过什么方法做成什么事情,对xxx的影响和意义是什么。

\subsubsection*{2.2论文的拟研究内容}

\begin{enumerate}
	\item[(1)] xxxxxxxx
	
	a)xxxxxxxxxxxx
	
    b)xxxxxxxxxxxx
	
	c)xxxxxxxxxxxx
	
	\item[(2)] xxxxxxxxxxx
	

	\item[(3)] xxxxxxxxxx
	



	\item[(4)] xxxxxxxxx
	

	
	\item[(5)] xxxxxxxxxxxx
	

\end{enumerate}


\subsubsection*{2.3拟解决的关键性问题}



\newpage
\section{学位论文研究计划及预期目标}
%\subsection*{1. 拟采取的主要理论、研究方法、技术路线和实施方案(可续页)}
\subsection*{3.1 拟采取的主要理论、研究方法、技术路线和实施方案}

\subsection*{3.1.1 研究方案(主要理论、研究方法、技术路线)}

\subsection*{3.1.2 实施方案}


\newpage
\subsection*{3.2 研究计划可行性,研究条件落实情况,可能存在的问题及解决办法(可续页)}

\subsubsection*{3.2.1 研究计划可行性}



\subsubsection*{3.2.2 已具备的实验条件主要包括:}



\subsubsection*{3.2.3 可能存在的问题与解决方法}

\newpage
\subsection*{3.3 研究计划及预期成果}


\begin{table}[h]
	\centering
	\renewcommand\arraystretch{1.5}
	
    \hrule height0pt \vfill
	\newlength{\tablesep}
	\setlength{\tablesep}{5pt}
	\newcolumntype{C}{>{\hfil}X<{\hfil}}
	\noindent\hskip-\npusep
	\begin{tabularx}{\npuwidth}{>{\centering}c|c|C}\hline
		\multirow{8}*{\parbox{0.05\npuwidth}{\centering  研\\究\\计\\划}}  & 起止年月 & 完成内容 
	\\ \cline{2-3} & 2021.03$ \sim $2021.09 &  xxxxxxxxxxxx
	\\ \cline{2-3} & 2021.03$ \sim $2021.09 &  xxxxxxxxxxxx
	\\ \cline{2-3} & 2021.03$ \sim $2021.09 &  xxxxxxxxxxxx
	\\ \cline{2-3} & 2021.03$ \sim $2021.09 &  xxxxxxxxxxxx
	\\ \cline{2-3} & 2021.03$ \sim $2021.09 &  xxxxxxxxxxxx
	\\ \cline{2-3} & 2021.03$ \sim $2021.09 &  xxxxxxxxxxxx
	\\ \cline{2-3} & 2021.03$ \sim $2021.09 &  xxxxxxxxxxxx
	\\ \hline
  \end{tabularx}
		
\end{table}


\noindent{\kaishu{预期创新点及成果形式}}


(1)预期创新点



(2)成果形成





\end{document}